\chapter{Input}

%%%%%%%%%%%%%%%%%%%%%%%%%%%%%%%%%%%%%%%%%%%%%%%%%%%%%%%%%%%%%%%%%%%%%%%%
%%%%%%%%%%%%%%%%%%%%%%%%%%%%%%%%%%%%%%%%%%%%%%%%%%%%%%%%%%%%%%%%%%%%%%%%
\section{Configuration file}
The configuration file is a fixed format file. Its name is the runnanme
given on the commandline, plus the ending ``.in''.
The first three lines are comments and ignored. The other lines have
one parameter per line with the parameter as first part, the rest is
ignored. The order is fixed. Below an example is given.
\begin{verbatim}
0.64                s                  - Radial coordinate s
0.036               M_t                - Mach number (for single Mach no. run)
1d0                 qi                 - Particle charge / elem. charge
2.014               mi                 - Particle mass / 1u
1d19                n0                 - particle density
40000000.0          vth                - thermal velocity
1.0                 epsmn              - perturbation amplitude B1/B0 (if pertfile==F)
0                   m                  - poroidal perturbation mode (if pertfile==F)
3                   n                  - toroidal perturbation mode (if pertfile==F, n>0!)
0                   mth                - poloidal canonical mode (for single Mach no. run)
10                  mthnum             - poloidal canonical mode count
-1.0d-1             Mtmin              - minimum Mach number
+1.0d-1             Mtmax              - maximum Mach number
21                  Mtnum              - Mach number count
F                   supban             - calculate superbanana plateau
T                   magdrift           - consider magnetic drift
F                   nopassing          - neglect passing particles
F                   calcflux           - calculate flux directly instead via flux-force relation
F                   noshear            - neglect magnetic shear term with dqds
F                   pertfile           - read perturbation from file with do_magfie_pert
F                   odeint             - use ODE integrator for resonance line
F                   nonlin             - do nonlinear calculation
1.0                 bscale             - scale B field by factor
1.0                 escale             - scale E field by factor
8                   inp_swi            - input switch for Boozer file
0                   orbit_mode_avg     - orbit mode for bounce averaging
0                   orbit_mode_transp  - orbit mode for transport computation
0                   vsteps             - integration steps in velocity space
F                   comptorque         - do torque computation
F                   intoutput          - output details of integrand
\end{verbatim}
The naming of the parameter, and the description is not required, but
for documentation purposes it is recomended to keep them.

\paragraph{s}
Radial coordinate of the flux surface. Note that figures from
publications might use inverse aspect ratio $\epsilon = 1/A$, with
aspect ratio $A$, as radial coordinate. In ``*__magfie_param.out'' all
three values are given (earlier versions might omit ``s'').

\paragraph{M\_t}
The toroidal Mach number. Is only used if ``Mtnum'' is set to $1$.

\paragraph{qi}
Particle charge in units of elementary charge.

\paragraph{mi}
Particle mass in dalton (approximately one proton mass).

\paragraph{n0}
Particle density.

\paragraph{vth}
Thermal velocity.

\paragraph{epsmn}
Perturbation amplitude $\epsilon_{M} = B1/B0$ (if pertfile==F).
Do not mix this up with inverse aspect ratio $\epsilon$.

\paragraph{m}
Poloidal perturbation mode (if pertfile==F)

\paragraph{n}
Toroidal perturbation mode (if pertfile==F, n>0!)

\paragraph{mth}
Poloidal canonical mode (for single Mach no. run)

\paragraph{mthnum}
Poloidal canonical mode count

\paragraph{Mtmin}
Minimum Mach number. See also ``Mtnum''.

\paragraph{Mtmax}
Maximum Mach number. See also ``Mtnum''.

\paragraph{Mtnum}
Mach number count. If equal to one, then the value from ``M\_t'' is
used. For larger values, an equidistant grid from ``Mtmin'' to ``Mtmax''
(including boundaries) is created, and the calculation is done for all
Mach numbers.

\paragraph{supban}
Calculate superbanana plateau.

\paragraph{magdrift}
Consider magnetic drift.

\paragraph{nopassing}
Neglect passing particles.

\paragraph{calcflux}
Calculate flux directly instead via flux-force relation.

\paragraph{noshear}
If true, then neglect magnetic shear term with dqds.

\paragraph{pertfile}
Read perturbation from file with do\_magfie\_pert.

\paragraph{odeint}
Use ODE integrator for resonance line.

\paragraph{nonlin}
Do nonlinear calculation.

\paragraph{bscale}
Scale B field by factor

\paragraph{escale}
Scale E field by factor.

\paragraph{inp\_swi}
Input switch for Boozer file.

\paragraph{orbit\_mode\_avg}
Orbit mode for bounce averaging.

\paragraph{orbit\_mode\_transp}
Orbit mode for transport computation.

\paragraph{vsteps}
If positive, then number of integration steps in velocity space for
midpoint rule. If zero, then an adaptive scheme is used.

\paragraph{comptorque}
Do torque computation.

\paragraph{intoutput}
Output details of integrand.

%%%%%%%%%%%%%%%%%%%%%%%%%%%%%%%%%%%%%%%%%%%%%%%%%%%%%%%%%%%%%%%%%%%%%%%%
%%%%%%%%%%%%%%%%%%%%%%%%%%%%%%%%%%%%%%%%%%%%%%%%%%%%%%%%%%%%%%%%%%%%%%%%
\section{plasma.in}

%%%%%%%%%%%%%%%%%%%%%%%%%%%%%%%%%%%%%%%%%%%%%%%%%%%%%%%%%%%%%%%%%%%%%%%%
%%%%%%%%%%%%%%%%%%%%%%%%%%%%%%%%%%%%%%%%%%%%%%%%%%%%%%%%%%%%%%%%%%%%%%%%
\section{profile.in}
