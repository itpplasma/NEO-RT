\section{Running \neort}

\subsection{Required files}

\paragraph{Parameter file}
A fixed format type file. First three lines (number fixed?) are
comments. Then at the beginning of each following line is one parameter.
It might be followed by a name and a description, but only he value is
relevant for reading by \neort.

There is a ``template'' form used for doing scans, e.g. over radius.

\paragraph{Boozer files for background field and perturbation}
Boozer type files, i.e. text files with magnetic field data in a
specific layout. Names are hardcoded(?) as ``in_file'' and
``in_file_pert''.

\paragraph{Profile file 1}
Text file with data in three columns. First is radial coordinate
(boozer s). Second is toroidal mach number. Third one (unused?) density? vth?

There is an octave script neo2\_to\_neort as part of \neotwo, to
create a \neort profile file from a \neotwo flux surface scan output.

\paragraph{Profile file 2}
Text file with special format (and fixed name 'plasma.in'?). First and
third line are comments (starting with '\%'), indicating quantities in the
following line(s). Second line contains five values, first is number of
flux surfaces, following two are masses (in units of proton mass) of
first and second ion species, respectivley. Forth and fifth values are
the corresponding charges (in elementary charges).
Starting in the fourth line, are the values for the quantities that are
assumed to vary across flux surfaces. The number of these lines should
match the value given in the second line.
Each of these lines has six colums. First column is flux surface label
(boozer s). Then follow density (in $1/m^3$) of first and second ion
species.
The last three rows are the temperatures (in $eV$) of the first species,
second species and of electrons, respectively.

\subsection{Running}
In case of a single run, simply execute \neort in the folder with the
input files.

For scans you can use the (executable) python3 script
run\_driftorbit.py, which requires first (inclusiv) and last (exclusive)
flux surface that should be used.

\subsection{output}
Note: for simplicity no flux surface numbering is done here. The number
would appear after the runname, e.g. 'runname42.out' or 'runname42_torque.out'.

\paragraph{runname.out}
Toroidal mach number and transport coefficients.
There are in total nine columns. First column is the toroidal mach
number. The other columns are parts/total of D11 and D12. Second column
gives part of D11 due to co-passing particles. Third column gives part
of D11 due to counter-passing particles. Fourth column gives part of D11
due to trapped particles. Fifth column is the sum of the three parts.
Columns six to nine are analogue for D12.

\paragraph{runname\_torque.out}
Boozer coordinate and

\subsection{Collecting Results}
Can be done with the (exectuable) python3 script
collect\_data\_from\_individual\_runs.py
